\input slides.tex
\input utf8-t1

\newpage %-------------------------------------------------------------------------------------------

\def\author{PPS}
\def\caption{PPXZ generator}
\def\date{24 Oct 2018}

\newpage %-------------------------------------------------------------------------------------------
\hbox{}
\vfil
\title{Toy simulation of $\rm p + p \rightarrow p + Z + X + p$ processes}
\vskip3mm
\centerline{N.~Turini, F.~Turini, L.~Marini, J.~Ka\v spar}
\vfil


\newpage %-------------------------------------------------------------------------------------------
\ctitle{Simulation package}{contents/sequence}

\> toy MC generator of $\rm p + p \rightarrow p + Z + X + p$
\>> X can be decayed to two particles
\>> Z can be decayed to a lepton pair

\> fast forward-proton simulation in PPS Roman Pots
\>> includes beam and vertex smearing at IP
\>> includes proton transport IP $\rightarrow$ RPs
\>> simulation of aperture limitations ($\xi$-$\th^*_x$ cut off deduced from LHC data)
\>> simulation of RecHits: with sensor acceptance and resolution

\> standard reconstruction RecHits $\rightarrow$ (lite) tracks

\> distribution plotter
\>> plots generator level quantities (X, Z, ...) with and without condition on protons reconstructed in RPs



\newpage %-------------------------------------------------------------------------------------------
\title{MC generator}

\> process $\rm p + p \rightarrow p + Z + X + p$ is generated as follows

\>> invariant mass of the X-Z system is generated according to exponential distribution
\vskip-3mm
\cThird
$$m_{\rm XZ} = m_{\rm XZ,min} + RandExponential(1/c)$$
\vskip2mm

\>> X and Z have isotropic momentum distributions in the c.m.s.~system of X-Z

\>> $p_z$ of the system of the 2 outgoing protons is generated with a uniform distribution
\cThird
\vskip-2mm
$$p_z(\hbox{2 protons}) = RandFlat(\hbox{min, max})$$
\vskip2mm

\>> mass of the Z generated as
\cThird
\vskip-2mm
$$m_Z = RandBreitWigner(\hbox{mean, gamma})$$
\vskip2mm

\>> if $m_{\rm XZ} < m_{\rm X} + m_{\rm Z}$, then generation repeated

\> X can be decayed to two particles
\>>> isotropic decay in X's rest frame

\> Z can be decayed to a pair of electrons or muons
\>>> isotropic decay in Z's rest frame


\newpage %-------------------------------------------------------------------------------------------
\ctitle{MC generator}{Default parameters}

\> mass of particle X: m\_X = $1200\un{GeV}$
\> masses of X decay products: m\_X\_pr1 = $100\un{GeV}$, m\_X\_pr2 = $0\un{GeV}$

\> mass of particle Z: m\_Z\_mean = $91.1876\un{GeV}$, m\_Z\_gamma = $2.4952\un{GeV}$
\> mass of electron: m\_e = $0.5109e-3\un{GeV}$
\> mass of electron: m\_mu = $105.658e-3\un{GeV}$
\> momentum of beam protons: p\_beam = $6500\un{GeV}$
\> generation of $m_{\rm XZ}$
\>> minimal value m\_XZ\_min = $1300\un{GeV}$
\>> parameter of exponential distribution: c = $0.04$
\> generation of $p_z(\hbox{2 protons})$
\>> mean: p\_z\_LAB\_2p\_min = $-1000\un{GeV}$
\>> sigma: p\_z\_LAB\_2p\_max = $+1500\un{GeV}$


\newpage %-------------------------------------------------------------------------------------------
\title{CMSSW implementation}

\> available in CMSSW branch:\\ \link{https://github.com/CTPPS/cmssw/tree/ctpps_initial_proton_reconstruction_CMSSW_10_2_0}{\SmallerFonts https://github.com/CTPPS/cmssw/tree/ctpps\_initial\_proton\_reconstruction\_CMSSW\_10\_2\_0}

\NormalFonts

\> modules
\>> MC generator:\\ SimCTPPS/Generators/plugins/PPXZGenerator.cc
\>> fast simulation:\\ SimCTPPS/OpticsParameterisation/plugins/CTPPSFastProtonSimulation.cc
\>> distribution plotter:\\ SimCTPPS/Generators/plugins/PPXZGeneratorValidation.cc

\> example config: SimCTPPS/Generators/test/test\_gen\_smear\_simu\_OF\_cfg.py
\>> 2017 layout of RPs (strips in 210-fr, pixels in 220-fr)
\>> preliminary 2017 geometry (RP position): $x_{\rm min}$ from data
\>> aperture limitations extracted from LHC data, preliminary tuning
\>> proton transport with 2017 optical functions (xangle = $140\un{\mu rad}$)
\>> realistic values for beam and vertex smearing


\newpage %-------------------------------------------------------------------------------------------
\title{Validation}

\> list MC validation checks
\>> total momentum of p + p + X + Z is zero and energy corresponds to $\sqrt s$
\>> total momentum of p + p + X + l + l is zero and energy corresponds to $\sqrt s$
\>> total momentum of p + p + Z + pr1 + pr2 (X decay products) is zero and energy corresponds to $\sqrt s$
\>> $p_{\rm T}$ distributions of X and Z compatible -- the XZ system is only boosted along z
\>> all distributions of positive and negative charge leptons from Z are compatible

\> plots on next slides
\>> obtained with the default parameters
\>> Z decayed to an electron pair
\>> colour code in 1D histograms:
\>>> blue: without requiring protons in RP acceptance
\>>> red: protons required in RP acceptance
\>>> RP acceptance defined: proton reconstructed in both arms, but in any RP (strip or pixel)


\newpage %-------------------------------------------------------------------------------------------
\title{Example hit distribution in RPs}

\centerline{\fig[13cm]{fig/hit_distribution.pdf}}


\newpage %-------------------------------------------------------------------------------------------
\title{Distributions of directly generated quantities}

\centerline{\fig[16cm]{fig/make_plots.pdf}}

\newpage %-------------------------------------------------------------------------------------------
\ctitle{Distributions}{$\xi_1$ vs.~$\xi_2$}

\> proton 1: plus side (LHC sector 45), proton 2: minus side (sector 56)

\centerline{\fig[16cm]{fig/xi2_vs_xi1.pdf}}


\newpage %-------------------------------------------------------------------------------------------
\ctitle{Distributions}{$p_{\rm T}$}

\centerline{\fig[16cm]{fig/p_T.pdf}}


\newpage %-------------------------------------------------------------------------------------------
\ctitle{Distributions}{$p_{z}$}

\centerline{\fig[16cm]{fig/p_z.pdf}}


\newpage %-------------------------------------------------------------------------------------------
\ctitle{Distributions}{$p$ (total)}

\centerline{\fig[16cm]{fig/p_tot.pdf}}


\newpage %-------------------------------------------------------------------------------------------
\ctitle{Distributions}{$\th$}

\centerline{\fig[16cm]{fig/theta.pdf}}


\newpage %-------------------------------------------------------------------------------------------
\ctitle{Distributions}{$\et$}

\centerline{\fig[16cm]{fig/eta.pdf}}

\vfil
\eject
\bye
